%
%
%

\documentclass{article}

\usepackage{a4wide}
\usepackage[latin1]{inputenc}

\parindent0em
\parskip1ex

\begin{document}

\newcommand{\ind}{$\;\;\;\;$}

%%%%%%%%%%%%%%%%%%%%%%%%%%%%%%%%%%%%%%%%%%%%%%%%%%%%%%%%%%%%%%%%%%%%%%%%%%%%%%%

\section{Software}

in pseudo -- c++ syntax

{\tiny

\fbox{\begin{tabular}{ll}
\multicolumn{2}{l}{CMOTOR}
\\
\hline private: & ENCODER\_STATE state
\\[1ex]
& POSIX\_CONDITION condition
\\[1ex]
& THREAD::Fxn ()\\
& \{\\
& \ind while (42)\\
& \ind \{\\
& \ind \ind state = BlockingReadFromMotor ()\\
& \ind \ind condition.Signal ()\\
& \ind \}\\
& \}
\\[1ex]
public: & SetEncoderVelocity ()
\\[1ex]
& ENCODER\_STATE GetEncoderState ()\\
& \{\\
& \ind condition.Wait ()\\
& \ind return state\\
& \}
\end{tabular}}

\fbox{\begin{tabular}{ll}
\multicolumn{2}{l}{CMOTORFEEDER}
\\
\hline private: & GetWorldState ()\\
& \{\\
& \ind CMOTOR::GetEncoderState ()\\
& \}
\\[1ex]
& THREAD::Fxn()\\
& \{\\
& \ind while (42) \{\\
& \ind \ind GetWorldState ()\\
& \ind \ind Localisation::updateOdometry ()\\
& \ind \}\\
& \}\\
public:\\
& SetWorldVelocity()\\
& \{\\
& \ind SetEncoderVelocity()\\
& \}\\
\end{tabular}}

} %\tiny

\subsection{Motoren}

\fbox{\begin{tabular}{ll}
\multicolumn{2}{l}{CMOTOR}
\\
\hline private: & ENCODER\_STATE state\\
& POSIX\_CONDITION condition\\
& THREAD::Fxn ()
\\
public: & SetEncoderVelocity ()\\
& GetEncoderState ()
\end{tabular}}

\fbox{\begin{tabular}{ll}
\multicolumn{2}{l}{CMOTORFEEDER}
\\
\hline private: & GetWorldState ()\\
& THREAD::Fxn()\\
public:\\
& SetWorldVelocity()\\
\end{tabular}}

%%%%%%%%%%%%%%%%%%%%%%%%%%%%%%%%%%%%%%%%

\subsection{Laserscanner}

\fbox{\begin{tabular}{ll}
\multicolumn{2}{l}{CLASER}
\\
\hline private: & CTHREAD::Fxn()
\\
public: & --
\end{tabular}}

\fbox{\begin{tabular}{ll}
\multicolumn{2}{l}{CLASERFEEDER}
\\
\hline protected: & CLASER *\_laser[]\\
& POSIX\_CONDITION condition\\
& Receive()\\
& CTHREAD::Fxn()
\\
public: & GetClosestObstacleDistance()\\
& GetLaserScan()
\end{tabular}}

%%%%%%%%%%%%%%%%%%%%%%%%%%%%%%%%%%%%%%%%

\subsection{Trajektoriengenerator}

\fbox{\begin{tabular}{ll}
\multicolumn{2}{l}{CPERIODICTIMER}
\\
\hline protecetd: & Fxn()\\
& virtual Event() = 0
\\
public: & --
\end{tabular}}

\fbox{\begin{tabular}{ll}
\multicolumn{2}{l}{CGENBASE : CPERIODICTIMER}
\\
\hline protected: & Localisation localisation\\
& CMOTORFEEDER motorFeeder\\
& CLASERFEEDER laserFeeder\\
& Fxn()\\
& virtual Event()
\\
public: & GetPos()\\
& SetPos()\\
& MotionRotate()\\
& MotionTranslate()
\end{tabular}}

\fbox{\begin{tabular}{ll}
\multicolumn{2}{l}{CMOTION}
\\
\hline private: & double \_transVel\\
& double \_rotVel\\
& virtual Step() = 0
\\
public: & GetVelocities()
\end{tabular}}

%%%%%%%%%%%%%%%%%%%%%%%%%%%%%%%%%%%%%%%%

\section{Zeitlicher Ablauf}

\subsection{New Setpoint for a Motion}

\begin{itemize}\itemsep0ex
\item \texttt{RTC} timer event occurs (every 31.25 milliseconds)
\item \texttt{CPERIODICTIMER::Fxn()::read("/dev/rtc")} returns
\item \texttt{CPERIODICTIMER::Fxn()} calls \texttt{CGENBASE::Event()}
\item \texttt{CGENBASE::Event()} calls \texttt{Localisation::predict()} for
last desired output
\item \texttt{CGENBASE::Event()} calls \texttt{MOTION::Step()} for next
setpoint
\item \texttt{CGENBASE::Event()} modifies setpoint for collision avoidance (if
appropriate)
\item \texttt{CGENBASE::Event()} computes inverse kinematic on setpoint
\item \texttt{CGENBASE::Event()} calls
\texttt{CMOTORFEEDER::SetWorldVelocity()}
\item \texttt{CMOTORFEEDER::SetWorldVelocity()} calls
\texttt{CMOTOR::SetEncoderVelocity()}
\item \texttt{CMOTOR::SetEncoderVelocity()} writes command via \texttt{RS232}
to motor
\item \texttt{CGENBASE::Event()} returns
\item \texttt{CPERIODICTIMER::Fxn()} calls \texttt{read("/dev/rtc")} to wait
for next event
\end{itemize}

\subsection{Reply from Motor}

\begin{itemize}\itemsep0ex
\item \texttt{RS232} data from motor becomes available
\item \texttt{CMOTOR::Fxn()::read("/dev/cua0")} returns available data
\item \texttt{CMOTOR::Fxn()} check completeness and sanity of reply
\item \texttt{CMOTOR::Fxn()} convert data to encoder state
\item \texttt{CMOTOR::Fxn()} update encoder state variable
\item \texttt{CMOTOR::Fxn()} signal condition (does not yet go to sleep?)
\item \texttt{CMOTOR::Fxn()::read()} waits for more data (goes to sleep)
\item \texttt{CMOTORFEEDER::Fxn()::GetWorldState()} wakes up on condition
\item \texttt{CMOTORFEEDER::Fxn()::GetWorldState()} converts encoder state to
world state (forward kinematic)
\item \texttt{CMOTORFEEDER::Fxn()::GetWorldState()} returns world state
\item \texttt{CMOTORFEEDER::Fxn()::updateOdometry()} calls Kalman filter
\item \texttt{CMOTORFEEDER::Fxn()::GetWorldState()} waits for next state update
\end{itemize}

\subsection{Laser Scanner}

note that there are two threads running two laser scanners in parallel. they're
synchronized by means of a posix condition. only if both threads have delivered
scan data within a certain amount of time are they used for positioning
(they're always used for collision avoidance).  for simplicity reasons the
following flowchart shows one laser thread only.

\begin{itemize}\itemsep0ex
\item \texttt{CLASER::Fxn()::read()} returns some data
\item \texttt{CLASER::Fxn()} check for completeness and sanity of data
\item \texttt{CLASER::Fxn()} calls \texttt{CLASERFEEDER::Receive()}
\item \texttt{CLASERFEEDER::Receive()} convert (radial) data to platform coordinates
\item \texttt{CLASERFEEDER::Receive()} detect reflector marks
\item \texttt{CLASERFEEDER::Receive()} signal condition (does not yet go to
sleep) and return
\item \texttt{CLASER::Fxn()} calls \texttt{read()} to wait for more data (goes
to sleep)
\item \texttt{CLASERFEEDER::Fxn()} wakes up on condition
\item \texttt{CLASERFEEDER::Fxn()} check if \textbf{both} scans are available
\item \texttt{CLASERFEEDER::Fxn()} calls \texttt{Localisation::updateLaser()}
on reflector marks
\item \texttt{CLASERFEEDER::Fxn()} wait on condition for new scans
\end{itemize}

%%%%%%%%%%%%%%%%%%%%%%%%%%%%%%%%%%%%%%%%%%%%%%%%%%%%%%%%%%%%%%%%%%%%%%%%%%%%%%%

\end{document}
